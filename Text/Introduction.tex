Работа посвящена решению задачи о движении волн на поверхности идеальной несжимаемой жидкости в однородном поле сил тяжести над подвижным и неподвижным дном раздичной формы. Влияние изменение дна на поведение поверхностных волн до сих пор недостаточно изучено, хотя его понимание является во многих случаях является важным. Для практики необходимо знать параметры, ответственные за механизмы возникновения и роста волн, законы их распространения, взаимодействия и действия на твердые тела.

Т.к. проведение натурных и лабораторных экспериментов в этой области почти невозможно, основным способом изучения данной проблемы является математическое моделирование, которое имеет ряд значимых преимуществ перед натурным экспериментом. В настоящее время содержание математического моделирования, его возможности и актуальность создания математических моделей претерпели крупные изменения. Это связано не только с быстрым развитием средств вычислительной техники, но и с совершенствованием существующих и разработкой новых численных методов реализации сложных математических моделей.

Существует несколько моделей, описывающих движение поверхностных волн разной степени сложности и адекватности реальному процессу их распространения. Но при использовании любой из них для описания движения волн в безграничной области существует проблема переноса краевых условий из бесконечности на границу расчетной области. 

В общем случае рассматриваемые процессы происходят на неограниченном участке пространства, но при моделировании необходимо поставить искусственные границы, чтобы предотвратить расчет бесконечного числа точек. При этом надо располагать границы достаточно далеко от исследуемого объекта, чтобы их влияние на интересующее нас ближнее поле было мало. Но такой подход применим не всегда.

Другой путь состоит в том, чтобы на близких искусственных границах ставить такие условия, которые не искажали бы поле внутри области, т.е. обеспечивали полное или частичное поглощение приходящих возмущений.

В данной работе рассматривается еще один способ ограничения расчетной области. Исследуется влияние изменения формы дна на поведение поверхностных волн с учетом переноса краевых условий из бесконечности на границу расчетной области на примере модели мелкой воды. Эта модель с достаточной точностью описывает движение волн цунами, т.к. предполагает, что высота волны много меньше ее длины. Уравнения, описывающие данную модель, используются в линейной и нелинейной форме. Полученные расчеты анализируются и сравниваются между собой.

Раздел <<Постановка задачи>> содержит математическую постановку и графическое описание задачи. В разделе <<Метод решения>> содержится алгоритм проведения расчетов, а также кратко описывается метод неполной аппроксимации. Графические результаты расчетов с указанием расчетных параметров помещены в раздел <<Результаты расчетов>>. В разделе <<Программные средства>> описывается техническая реализация расчетной программы и используемые сторонние средства.
